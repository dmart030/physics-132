\documentclass[12pt]{report}
\usepackage{graphicx}
\usepackage{booktabs}
\graphicspath{ {images/} }
\usepackage[english]{babel}
\usepackage{amsmath}
\begin{document}
\title{Lab7 \\ Latent heat of Fusion of Ice \\ Physics 132 Lab}
\author{Dominic Martinez-Ta}
\date{22 May 2015}
\maketitle


\section{Purpose}
	
	This experiment makes use of measuring the heat capacity of a calorimeter in order to improve the accuracy of a measurement of the latent heat of fusion of ice. In such a mueasurement, the heat transfers between ice, water, the calorimeter and the environment are given by:


\begin{equation}
Q_{water}(T_i \rightarrow T_f) = m_{water}\int_{T_i}^{T_f} c_p(T) dT
\end{equation}

\begin{equation}
Q_{cal}(T_i \rightarrow T_f) = \int_{T_i}^{T_f} C_{cal}(T) dT
\end{equation}

\begin{equation}
Q_{env}(T_i \rightarrow T_f) = \frac{Q_{cal}(T_i \rightarrow T_f)}{T_f - T_i} \frac{dT}{dt} \Delta t
\end{equation}

\begin{equation}
Q_{ice}(T_f) = m_{ice}L + m_{ice}\int_{0^{\circ} C}^{T_f} c_p(T) dT
\end{equation}


And by conservation of energy: 

\begin{equation}
\Delta U (T_i \rightarrow T_f) = Q_{water}(T_i \rightarrow T_f) + Q_{cal}(T_i \rightarrow T_f) + Q_{ice}(T_i \rightarrow T_f) = 0
\end{equation}

\section {Materials and resources}

\begin{itemize}
	\item Calorimeter
	\item Digital scale
	\item Hot \& cold water
	\item ice
	\item two 250 ML beakers
	\item Temperature sensor
	\item Data Studios
	\item \textbf{Love}
\end{itemize}

\section{Procedure}

\textbf{Part I: Heat Capacity and the Calorimeter}\\
The heat energy used to changef the temperature of the inner wall of the thermos is charactized by
	\[mc_p(T_2 - T_f = C_{cal}(T_f - T_1)\]

\begin{enumerate}
	\item Add 250 mL of cold water into the calorimeter. Allow several minutes for it to reach equilibirum.
	\item Measure the common temperature of teh water and the calorimeter. (This will be $T_1$).
	\item pour out the water and measure the weight of the empty beaker.
	\item Get 200 mL of hot water in the beaker.
	\item Measure the temperature of the hot water in the beaker. Take 30 temperature readings in 30 seconds, and record the 30th reading as $T_2$
	\item Repeat this measurement two more times.
	\item From the average of the three trials, calculate the uncertainty in L, only once, given by:
		\[ \Delta C_{cal} = C_{cal}\sqrt{ (\frac{\Delta m_{water}}{m_{water}})^2 + ( \frac{\Delta T_{cal}}{T_{cal}})^2} \]
\end{enumerate}
\textbf{Part II: Latent Heat of Fusion of Ice}
\begin{enumerate}
	\item Weigh the calorimeter plus the hot water together.
	\item Record the initial temperature of the water and calorimeter, $T_1$. While gently stirring the water, record the
temperature at 1 second intervals for 6 minutes, take measurements for about 1 minute and then proceed onto
step number three. (Continue to take data and generate a graph of temperature vs. time.)
	\item Carefully dry the ice cube in the Calorimeter. (\textbf{HOW IS THAT POSSIBLE?!!})
	\item The observed T vs. t graph should contain three distinct regimes: the initial constant temperature, a
drastic drop in temperature, and the final temperature. Record a copy of T vs. t in the lab report
	\item The temperature after the initial exchange of heat is the final temperature of the water, melted ice, and the
calorimeter, $T_f$
	\item Weigh the calorimeter with the water and melted ice. Calculate the mass of the ice from this measure-ment
and the previous measurement of the calorimeter and original water alone.
	\item Repeat this measurement two more times, and take the average of the three trials for lab report calculations.
	\item From the average of the three trials, calculate the uncertainty in L, only once, given by: 
		\[ \Delta L = L\sqrt{ [\delta (T_f - 271.15K)]^2 + [\delta (\Delta T)]^2 + [\frac{ \Delta C_{cal}}{C_{cal}}}]^2\]
\end{enumerate}



\section{Data Analysis}
	
\section{Questions}

\section{Answers to questions}

\section{Conclusion}

\section{Data}

\begin{center}

\begin{table}[h]
\textbf{Part I}, chromium \\
 $\begin{array}{ *{7}{c} }
\toprule
T_H(C) & T_C(C) & T_F(C) & M_{Total}(g) &  m_{H_2O}(g) & m_{cup}(g) & m_{chromium}(g) \\
\midrule
77.0 & 63.7 & 59.7 & 145.73 & 136.22 & 2.00 & 7.51 \\ 
65.7 & 49.9 & 47.6 & 125.88 & 115.37 & - & -\\ 
52.5 & 33.8 & 31.6 & 125.76 & 116.25 & - & - \\ 
34.6 & 18.6 & 18.2 & 126.42 & 116.91 & - & - \\ 
21.4 & 4.2 & 3.1 & 122.34 & 112.83 & - & - \\ 
\end{array}$
\end{table}
\end{center}
\begin{center}
\begin{table}[h]
\textbf{Part I}, Copper \\
 $\begin{array}{ *{7}{c} }
\toprule
T_H(C) & T_C(C) & T_F(C) & M_{Total}(g) &  m_{H_2O}(g) & m_{cup}(g) & m_{Cu}(g) \\
\midrule
77.0 & 60.0 & 61.2 & 142.00 & 100.00 & 2.00 & 40.00 \\ 
65.7 & 50.0 & 52.1 &  148.00 & 106.00 & - & - \\ 
48.6 &  34.7 & 34.7 & 153.30 & 113.30 & - & - \\ 
35.0 &  21.0 &  21.0 &  134.43 &  92.43 & - & - \\ 
21.0 & 5.0 &  7.3 &  95.7 & 53.7 & - & - \\ 
 \end{array}$
\end{table}
\begin{table}[h]
\textbf{Part II}, Chromium \\
 $\begin{array}{ *{9}{c} }
\toprule
T_H(C) & T_C(C) & T_F(C) & M_{Total}(g) &  m_{eth}(g) & m_{cup}(g) & m_{chromium}(g)  & \frac{\partial m}{\partial t} & \Delta t(s)\\
5.0 & -10.0 & -7.3 & 40.2 & 10.0 & 2.0 &  7.51 & 0.02 & 7 \\
-10.11 & -25 & -24.41 & 38.5 & - & - & - & 0.01 & 8 \\
-24.12 & -40.0 & -41.05 & 40.5 & - & - & - & 0.03 & 3 \\
-40.00 & -55.0 & -54.45 & 44.6 & - & - & - & 0.02 & 5 \\
-55.7 &  -75.0& -74.45 & 38.2 & - & - & - & 0.08 & 13 \\
\midrule
\end{array}$
\end{table}
\begin{table}[h]
\textbf{Part II}, Copper \\
 $\begin{array}{ *{9}{c} }
\toprule
T_H(C) & T_C(C) & T_F(C) & M_{Total}(g) &  m_{eth}(g) & m_{cup}(g) & m_{Cu}(g)  & \frac{\partial m}{\partial t} & \Delta t(s)\\
5.0 & -9.8 & -7.3 & 40.2 & 10.0 & 2.0 &  40.0 & 0.13 & 10 \\
-10.2 & -25.4 & -24.41 & 38.5 & - & - & - & 0.15 & 9 \\
-22.5 & -39.9 & -41.05 & 40.5 & - & - & - & 0.09 & 9 \\
-42.3 & -56.2 & -54.45 & 44.6 & - & - & - & 0.15 & 11 \\
-55.2 &  -75.5 & -74.48 & 38.2 & - & - & - & 0.07 & 13 \\
\midrule
\end{array}$
\end{table}
\end{center}
\end{document}